\section{Conclusion}

Designing security measures for an on-demand car access system is not trivial. Different considerations have been taken into account to design an application that is both easy to use for users and secure (against different kinds of attacks). The application has been designed for strong 128-bit security, however due to the weaker RSA-keys used in the Belgian eID, the final application only has 112-bit security. If the application were to be deployed in countries where this technology has already advanced (e.g.\ Germany), true 128-bit security will be achieved. Nonetheless, the 112-bit security should result in sufficient security strength until 2030, thus fulfilling the requirement to implement a system that is robust in the short and mid-term. To conclude, the cryptographic schemes were designed with embedded hardware in mind (hence the extensive use of ECC) which should make converting these schemes for other applications easy.

The prototype enables us to illustrate the basic cryptographic functions and schemes used in the Car~Key Application. It is sufficiently complete yet remains a simple proof of concept to showcase how the application would behave without including all technical details involved.
